\documentclass{imposter}
%\documentclass[landscape]{imposter}

\usepackage{layout}

\title{Some sort of ocean model with a long name}
\author{David A. Ham}
\department{Department of Mathematics}
\institution{Imperial College London}
\email{David@ham.dropbear.id.au}
\url{http://amcg.ese.ic.ac.uk}

%% Primary university logo.
%\logo{imperial.eps}
% Secondary logos (not supported in all poster styles)
%\logoa{AMCG.eps}

%% Options to set colors of multiple poster features.
%\titlecolor{named}{Blue}
%\postercolor{named}{Red}
%\pagecolor{green}

\posterstyle{mpecdtstyle}

% This defines the command \showgrid. Use this to get a background grid to
% your poster text area. This helps when positioning poster contents with
% \rput.
\newpsobject{showgrid}{psgrid}{subgriddiv=1,griddots=10,gridlabels=20pt}

% This sets up boxes to have nice rounded corners and somewhat thicker lines
% than the default.
\psset{linecolor=postercolor,linewidth=.2,cornersize=absolute,linearc=0.5,framesep=0.5}

\begin{document}
\begin{pspicture}(0,0)(82,96)
%\showgrid
  \rput[tl](0,96){\psframebox{\begin{minipage}{37cm}
        \section{Finite elements for geophysics}

        Geophysical flows are are characterised by, among other features,
        high aspect ratios, geostrophic balance and the importance of waves
        as information vectors. We have previously shown that $\textsf{P1}_{\textsf{DG}}$-$\textsf{P2}$
        element, which we designed for geophysical applications, has:
        \begin{itemize}
        \item \textbf{exactly stable geostrophically balanced states.}
        \item \textbf{no spurious pressure modes.}
        \end{itemize}
        Here we demonstrate another geophysically desirable property:
        \begin{itemize}
        \item \textbf{optimally convergent inertia-gravity wave solutions.}
        \end{itemize}

        \end{minipage}
      }
    }

\end{pspicture}


\end{document}
