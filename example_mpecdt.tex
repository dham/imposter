\documentclass{imposter}
%\documentclass[landscape]{imposter}

\usepackage{listings}

\title{Imposter, a \LaTeX\ poster class for Imperial College London}
\author{David A. Ham}
\department{Department of Mathematics}
\institution{Imperial College London}

%% Primary university logo.
%\logo{imperial.eps}
% Secondary logos (not supported in all poster styles)
%\logoa{AMCG.eps}

%% Options to set colors of multiple poster features.
%\titlecolor{named}{Blue}
%\postercolor{named}{Red}
%\pagecolor{green}

\posterstyle{mpecdtstyle}


% This sets up boxes to have nice rounded corners and somewhat thicker lines
% than the default.
\psset{linecolor=postercolor,linewidth=.2,cornersize=absolute,linearc=0.5,framesep=0.5}



\begin{document}
%\showgrid
  \leftbox{96}{
        \section{Getting started}
        
        You should use the file \texttt{example\_mpecdt.tex} as a template. The
        easiest thing to do is to make a copy of this file. 

        If you have installed Imposter in accordance with the instructions
        then the template can be copied to the current directory as follows:

        \begin{description}
        \item[Ubuntu (installation using apt-get)] \mbox{}\\
            \texttt{\footnotesize cp /usr/share/texmf/doc/latex/imposter/example\_mpecdt.tex .}          

        \item[Other Mac or Linux systems] \mbox{}\\
            \texttt{\footnotesize cp \$(kpsewhich --var-value
              TEXMFHOME)/doc/latex/imposter/example\_mpecdt.tex .}
        \end{description}
        Don't forget the final full stop when copying!
        \subsection{Portrait or landscape}

        The first decision you need to make, since it will affect the rest
        of the layout, is whether the poster is to be portrait or
        landscape. Check the instructions provided by the conference at
        which you will present. The default is portrait; landscape can be
        selected by passing the optional parameter \texttt{landscape}\ to
        the class:\\

        \texttt{\textbackslash documentclass[landscape]\{imposter\}}
    }
  \leftbox{59}{
      \section{Positioning boxes}
      \LaTeX\ is in many respects not the ideal platform for
      posters. Posters are a visual medium in which the author will want to
      control the position of the contents. \LaTeX, on the other hand is
      based on the concept that the author describes the content and the
      layout is handled separately. The result of this is that you'll need
      to do a little more manual position of content than you would when
      writing a paper.\\

      Imposter employs the \LaTeX\ package \texttt{pstricks}\ to enable
      fine-scale positioning. Pstricks provides commands which position a
      box relative to a pstricks picture environment, and Imposter provides
      some custom box commands for the common cases:\\

      \texttt{\textbackslash leftbox\{height\}\{contents\}}\\

      A framed box on the left hand side of the poster whose top is
      \texttt{height}\ centimeters from the bottom of the poster area.\\

      \texttt{\textbackslash rightbox\{height\}\{contents\}}\\
      
      A framed box on the right hand side of the poster whose top is
      \texttt{height}\ centimeters from the bottom of the text area.\\

      In landscape mode, a third command is available:\\

      \texttt{\textbackslash centrebox\{height\}\{contents\}}\\

      Which creates a box in the centre columm.\\

      In portrait mode, the top box in each column should start at a height
      of 96. In landscape mode, the start height is 61.
    }
%Also do graphics.
    \rightbox{96}{
        \section{Don't use pdflatex!}
        
        Because Imposter uses pstricks, \texttt{pdflatex}\ won't correctly
        compile your poster. Instead, you need to use plain \LaTeX\
        and \texttt{dvipdf}. For example:\\

        \texttt{latex example\_mpecdt.tex}\\
        \texttt{dvipdf example\_mpecdt.dvi} 

        \subsection{eps graphics only}
        
        A consequence of using plain \LaTeX\ is that encapsulated postscript
        (eps) is the only supported format for graphics. Fortunately, other
        graphics formats can usually be losslessly converted to eps. For
        example, pdf graphics can be converted using \texttt{pdftoeps}.
    }


\end{document}
